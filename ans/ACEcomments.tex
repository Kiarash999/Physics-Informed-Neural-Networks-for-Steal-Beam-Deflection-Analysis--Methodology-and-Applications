\documentclass{ar2rc}
\usepackage{hyphenat}
% \hyphenpenalty=10000
% \exhyphenpenalty=10000



\title{Physics-Informed Neural Networks for Beam Deflection Analysis: Methodology and Applications}
\author{K. Baharan, H. Mirzabozorg}
\journal{Advanced Computational Engineering}
\doi{}

\begin{document}
	
	\maketitle

%%%%%%%%%%%%%%%%%%%%%%%%%%%%%%%%%%%%%%%%%%%%%%%%%%%%%%%%%%%%%%%%%%%%%%%%%%%%%%%%%%
%%%%%%%%%%%%%%%%%%%%%%%%%%%%%%%%%%%%%%%%%%%%%%%%%%%%%%%%%%%%%%%%%%%%%%%%%%%%%%%%%%
%%%%%%%%%%%%%%%%%%%%%%%%%%%%%%%%%%%%%%%%%%%%%%%%%%%%%%%%%%%%%%%%%%%%%%%%%%%%%%%%%%
%%%%%%%%%%%%%%%%%%%%%%%%%%%%%%%%%%%%%%%%%%%%%%%%%%%%%%%%%%%%%%%%%%%%%%%%%%%%%%%%%%
%%%%%%%%%%%%%%%%%%%%%%%%%%%%%%%%%%%%%%%%%%%%%%%%%%%%%%%%%%%%%%%%%%%%%%%%%%%%%%%%%%

	\section{Reviewer \#1}



%--------------------------------kiarash answer----------------------------------%
	\subsection{Comment 1}
	\RC\textbf{Boundary-condition transform: scope \& proof.}\\
    The hard constraint $x(1-x)$ enforces $w(0)=w(L)=0$, but the text also claims exact satisfaction for fourth-order systems. Please clarify what is enforced exactly: displacement only, or slope/moment/shear as well? Provide derivations/examples for mixed BCs (e.g., clamped-free, clamped-pinned). If the method targets only Dirichlet displacement BCs, state limits explicitly.
	
  \AR\ We appreciate the reviewer's insightful comment. The hard-constrained transformation 
  \[
  w_\theta(x) = x(1-x)\,\mathrm{NN}(x)
  \]
  enforces only the \emph{Dirichlet displacement boundary conditions}, i.e., $w(0)=w(L)=0$, exactly. The higher-order boundary quantities (slope, bending moment, and shear) are not automatically satisfied by this transformation and are instead imposed through the physics-informed loss terms derived from the governing differential equation and corresponding natural boundary conditions.

  To clarify, for a fourth-order beam equation, the boundary conditions can be of mixed types:
  \begin{itemize}
    \item \textbf{Simply supported beam:} $w(0)=w(L)=0$ (enforced exactly by the transform) and $M(0)=M(L)=0$ (enforced weakly via loss).
    \item \textbf{Clamped-free beam:} $w(0)=w'(0)=0$ (partially enforced using a modified transform, e.g. $x^2(3-2x)\mathrm{NN}(x)$) and $M(L)=V(L)=0$ (included in the residual loss).
    \item \textbf{Clamped-pinned beam:} $w(0)=w'(0)=0$, $w(L)=0$, $M(L)=0$ — a mixed case where $x(1-x)$ cannot fully encode all constraints and thus requires combined hard-soft enforcement.
  \end{itemize}

  Accordingly, we have revised the text to state explicitly that the proposed $x(1-x)$ transformation ensures exact satisfaction of displacement boundary conditions only. Other boundary quantities are treated as soft constraints through loss terms, which maintain high accuracy while preserving training flexibility.



%--------------------------------kiarash answer----------------------------------%
	\subsection{Comment 2}
	
	\RC\textbf{Adaptive weighting ablation.}\\
    Support the scheduler $W_{BC}(t)=10e^{-0.0001t}$ with ablation: constant vs.exponential, different initial weights/decays, and report sensitivity. Normalize loss terms (e.g., by BC/PDE residual magnitudes) to ensure the schedule is not just rescaling.
	
  \AR\ In response to the reviewer's request on adaptive weighting, we trained a direct unweighted baseline (no exponential decay; constant loss weights). Results show that without the adaptive boundary-weight scheduler the model fails to converge to the high-precision solution: final total loss increased from $1.84 \times 10^{-4}$ to $1.81 \times 10^{9}$ and the relative L2 error increased from 0.056\% to $6.95 \times 10^{3}$. This demonstrates the necessity of the proposed scheduler under our training protocol. A full ablation sweep (varying initial weights, decay rates and performing loss-term normalization) can be provided upon request.



%------------------------------professor answer----------------------------------%
	\subsection{Comment 3}
	
	\RC\textbf{Gaussian regularization for point load.}\\
    You cite rules for $\sigma$ (function of $L$ and collocation density $N_c$) but then fix $\sigma=0.01L$. Provide a systematic study of $\sigma$: error vs.\ $\sigma$, and how it affects recovering the expected jump in shear/continuity in $w^{(3)}$. Report moment/shear fields and verify jump conditions at $x=L/2$.
	
  \AR nan 


%%%%%%%%%%%%%%%%%%%%%%%%%%%%%%%%%%%%%%%%%%%%%%%%%%%%%%%%%%%%%%%%%%%%%%%%%%%%%%%%%%
	\subsection{Comment 4}
	\RC\textbf{Quantitative evaluation \& fairness of baselines.}\\
    Provide tables of relative $L_2$ error and BC max violation for all cases, not only a single point-load. Compare against: (a) classical FEM with multiple mesh sizes; (b) a soft-constraint PINN; (c) a PINN with Fourier features/RAS as cited. Include wall-clock time and hardware; "5x speedup" needs evidence and precise conditions.

	\AR nan

%%%%%%%%%%%%%%%%%%%%%%%%%%%%%%%%%%%%%%%%%%%%%%%%%%%%%%%%%%%%%%%%%%%%%%%%%%%%%%%%%%
	\subsection{Comment 5}
	\RC\textbf{Convergence claims vs. solution accuracy.}\\
    Training loss values (e.g., $\mathcal{O}(10^{-10})$) do not necessarily reflect solution error. Plot validation error vs.\ iterations and show early-stopping behavior.

	\AR	nan

%%%%%%%%%%%%%%%%%%%%%%%%%%%%%%%%%%%%%%%%%%%%%%%%%%%%%%%%%%%%%%%%%%%%%%%%%%%%%%%%%%
	\subsection{Comment 6}
	\RC\textbf{Reproducibility package.}\\
    Report multiple random seeds with mean~$\pm$~std to demonstrate robustness. List all hyperparameters (depth/width, activation, optimizer, LR schedule, batch/collocation sizes, BC point sampling, PDE residual weighting, L-BFGS settings, gradient clipping, random seeds). Provide exact material/geometry parameters used (E, I, P, q, L, units) and the non-dimensionalization if any. If code is shared, ensure it's anonymized for double-blind review (the current GitHub mention would break anonymity).
    
	\AR\ We have added a reproducibility table to the manuscript and provided an anonymized code archive in the supplementary materials. All reported performance numbers are computed as mean~$\pm$~std across multiple random seeds.
  

%%%%%%%%%%%%%%%%%%%%%%%%%%%%%%%%%%%%%%%%%%%%%%%%%%%%%%%%%%%%%%%%%%%%%%%%%%%%%%%%%%
  \subsection{Comment 7}
  \RC\textbf{Higher-order derivatives \& smoothness.}\\
    Since Euler-Bernoulli uses $w^{(4)}$, discuss numerical stability of AD on deep networks and whether Sobolev training/gradient regularization or swish vs.\ tanh materially changes higher-order derivative quality.
	
  \AR nan

%%%%%%%%%%%%%%%%%%%%%%%%%%%%%%%%%%%%%%%%%%%%%%%%%%%%%%%%%%%%%%%%%%%%%%%%%%%%%%%%%%
  \subsection*{Summary}
  The manuscript proposes three PINN enhancements for Euler--Bernoulli beam deflection problems:
  \begin{enumerate}
      \item a hard-constrained output transform $w_\theta(x)=x(1-x)\,\mathrm{NN}(x)$ to enforce fixed-end Dirichlet BCs;
      \item an exponentially-decaying boundary-loss weight $W_{BC}(t)=10e^{-0.0001t}$; and
      \item a Gaussian regularization of Dirac delta point loads with $\sigma=0.01L$.
  \end{enumerate}
  Results are shown for cantilever, fully restrained, and mid-span point-loaded beams, with training curves and brief accuracy claims.


%%%%%%%%%%%%%%%%%%%%%%%%%%%%%%%%%%%%%%%%%%%%%%%%%%%%%%%%%%%%%%%%%%%%%%%%%%%%%%%%%%
%%%%%%%%%%%%%%%%%%%%%%%%%%%%%%%%%%%%%%%%%%%%%%%%%%%%%%%%%%%%%%%%%%%%%%%%%%%%%%%%%%
%%%%%%%%%%%%%%%%%%%%%%%%%%%%%%%%%%%%%%%%%%%%%%%%%%%%%%%%%%%%%%%%%%%%%%%%%%%%%%%%%%
%%%%%%%%%%%%%%%%%%%%%%%%%%%%%%%%%%%%%%%%%%%%%%%%%%%%%%%%%%%%%%%%%%%%%%%%%%%%%%%%%%
%%%%%%%%%%%%%%%%%%%%%%%%%%%%%%%%%%%%%%%%%%%%%%%%%%%%%%%%%%%%%%%%%%%%%%%%%%%%%%%%%%

	\newpage
	\section{Reviewer \#2}

  \subsection{Comment 1}

	\RC\textbf{Novelty and Comparative Assessment:} The claim of a 37\% convergence improvement via adaptive weighting—how was this quantified? Was it based on a direct comparison with a static weighting scheme under identical conditions (network architecture, optimizer, initializations)? Please provide more details or a side-by-side comparison figure/table.
	
  \AR nan

%%%%%%%%%%%%%%%%%%%%%%%%%%%%%%%%%%%%%%%%%%%%%%%%%%%%%%%%%%%%%%%%%%%%%%%%%%%%%%%%%%
  \subsection{Comment 2}

	\RC\textbf{Implementation Details and Reproducibility:}
    \begin{itemize}
        \item The GitHub link provided in the Limitations section appears to be broken. Please ensure the code is publicly available and the link is correct for the sake of reproducibility.
        \item What was the exact number and spatial distribution of collocation points ($N_c$) for each case? Was uniform random sampling used, or a different strategy?
    \end{itemize}
	
  \AR nan

%%%%%%%%%%%%%%%%%%%%%%%%%%%%%%%%%%%%%%%%%%%%%%%%%%%%%%%%%%%%%%%%%%%%%%%%%%%%%%%%%%
  \subsection{Comment 3}

	\RC\textbf{Error Analysis and Validation:}
    \begin{itemize}
        \item Why is the relative $L_2$ error for the point load case (0.56\%) higher than the 0.30\% reported by Zhang et al.\ (2020)? Does this indicate a limitation of the Gaussian regularization approach compared to other singularity-handling methods?
        \item Was the Maximum Absolute Error computed for all cases? It is only reported in the comparative Table 6. Presenting it for all case studies would provide a more complete picture of solution accuracy.
    \end{itemize}

	\AR nan

%%%%%%%%%%%%%%%%%%%%%%%%%%%%%%%%%%%%%%%%%%%%%%%%%%%%%%%%%%%%%%%%%%%%%%%%%%%%%%%%%%
  \subsection{Comment 4}
	
	\RC\textbf{Parameter Selection and Sensitivity:}
    \begin{itemize}
        \item Was a systematic sensitivity analysis performed on key hyperparameters, such as the Gaussian bandwidth ($\sigma$), the decay rate in $W_{BC}(t)$, or the network depth/width? The choice of $\sigma=0.01L$ is stated as optimal, but the process for determining this should be explained (e.g., via a brief parametric study).
        \item \textbf{Material Property:} The material properties are defined only via the composite parameter $EI$. Please clarify: What specific material is being modeled? (e.g., structural steel with $E=200$ GPa, or a generic material?). This is crucial for readers to contextualize the physical scale of the deflections and the value of $EI=200\ \mathrm{N\cdot m^2}$ used in the point load example, which seems unusually low for real-world beams (suggesting a lab-scale or normalized example).
    \end{itemize}

	\AR nan

%%%%%%%%%%%%%%%%%%%%%%%%%%%%%%%%%%%%%%%%%%%%%%%%%%%%%%%%%%%%%%%%%%%%%%%%%%%%%%%%%%
  \subsection{Comment 5}
	
	\RC\textbf{Scalability and Generalizability:}
    \begin{itemize}
        \item Has the proposed framework been tested on more complex boundary conditions (e.g., multi-span beams, springs) or time-dependent loads?
        \item What are the anticipated primary challenges in extending this 1D methodology to 2D plate or 3D solid mechanics problems, beyond those mentioned in the limitations?
    \end{itemize}

	\AR nan

%%%%%%%%%%%%%%%%%%%%%%%%%%%%%%%%%%%%%%%%%%%%%%%%%%%%%%%%%%%%%%%%%%%%%%%%%%%%%%%%%%
	\subsection{Comment 6}

  \RC\textbf{Convergence Analysis:}
    \begin{itemize}
        \item Was the identified three-phase convergence pattern (boundary fitting, physics compliance, fine-tuning) consistent across all three benchmark problems? Can this behavior be justified analytically or linked to the properties of the optimizer?
    \end{itemize}

  \AR nan

%%%%%%%%%%%%%%%%%%%%%%%%%%%%%%%%%%%%%%%%%%%%%%%%%%%%%%%%%%%%%%%%%%%%%%%%%%%%%%%%%%
	\subsection{Comment 6}

  \RC\textbf{Comparison with Classical Methods:}
    \begin{itemize}
        \item Beyond setup time, was a direct comparison of computational runtime and memory usage performed between the proposed PINN and a standard FEM solver (e.g., Abaqus or FEniCS) for an equivalent level of accuracy?
    \end{itemize}

  \AR nan


	\end{document}