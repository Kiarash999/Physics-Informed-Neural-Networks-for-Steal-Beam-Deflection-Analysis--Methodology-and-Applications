\documentclass[sn-apa]{sn-jnl} % Springer Nature journal style

% --- Packages ---
\usepackage{tikz}
\usepackage{amsmath, amssymb}
\usepackage{graphicx}
\usepackage{subcaption}
\usepackage{booktabs}
\usepackage{physics}
\usepackage{pgfplots}
\pgfplotsset{compat=1.18}
\usepackage{hyperref}
\hypersetup{
    colorlinks = true,
    linkcolor = blue,
    citecolor = blue,
    urlcolor = blue
}
\usepackage{siunitx}

\jyear{2025}

% --- Title & Author ---
\title[PINNs for Beam Deflection]{Physics-Informed Neural Networks for Beam Deflection Analysis: Methodology and Applications}

\author*[1]{\fnm{Kiarash} \sur{Baharan}}
\author[2]{\fnm{Coauthor} \sur{Name}}
\affil[1]{\orgdiv{Department of Civil Engineering}, \orgname{Azad University Najafabad}, \orgaddress{\city{Isfahan}, \country{Iran}}}
\affil[2]{\orgdiv{Department}, \orgname{Institution}, \orgaddress{\city{City}, \country{Country}}}

\abstract{
This paper presents an advanced Physics-Informed Neural Network (PINN) methodology for structural beam analysis, introducing three key methodological improvements that address computational challenges in mechanics. We develop: (1) A hard-constrained architecture through output transformation $w_{\theta}(x) = x(1-x)\cdot\textrm{NN}(x)$ that enforces boundary conditions exactly for fourth-order systems; (2) A dynamic weighting scheme $w_{\text{BC}}(t)=10\cdot\exp(-0.0001t)$ that adaptively balances boundary constraints and PDE residuals during training, improving convergence efficiency; and (3) A regularized formulation using Gaussian approximation ($\sigma = 0.01L$) for Dirac delta loads that handles singularities without domain decomposition. Validated on cantilever, fully-restrained, and point-loaded beams, our approach achieves high accuracy ($\mathcal{O}(10^{-10})$ loss) with 0.56\% relative L2 error for concentrated mid-span loads. The method demonstrates computational advantages including mesh-independent analysis and reveals distinct convergence phases through comprehensive training dynamics. This work establishes PINNs as a viable alternative for structural deflection analysis with extensions to inverse problems.
}

\keywords{Physics-Informed Neural Networks, Beam Deflection, Boundary Condition Enforcement, Gaussian Regularization, Structural Mechanics}

\maketitle

\section{Introduction}
\section{Introduction}
Physics-Informed Neural Networks (PINNs) have emerged as a transformative approach in computational mechanics, bridging the gap between data-driven learning and physics-based modeling. By embedding governing differential equations directly into the training process of neural networks, PINNs circumvent the need for mesh generation and traditional discretization, offering a mesh-free framework for solving boundary value problems. This capability is particularly advantageous in structural engineering, where complex geometries, high-order partial differential equations (PDEs), and singular loading conditions often pose challenges to conventional numerical methods.

In recent years, significant attention has been devoted to advancing PINN methodologies for structural applications, including beams, plates, and frame systems. Despite notable successes, challenges persist in achieving rapid convergence, handling singularities such as point loads, and enforcing high-order boundary conditions with precision. Addressing these issues is essential for expanding the practical utility of PINNs in engineering analysis and design.

This study focuses on the Euler–Bernoulli beam model as a representative testbed for developing and validating methodological innovations in PINNs. We introduce three key contributions: (1) a hard-constrained output transformation that enforces fourth-order boundary conditions exactly; (2) an exponentially decaying adaptive weighting strategy to balance boundary and PDE residual losses throughout training; and (3) a Gaussian-regularized Dirac delta approximation to model concentrated loads without resorting to domain decomposition. Through extensive numerical experiments on cantilever, fully restrained, and point-loaded beams, we demonstrate that these advancements significantly enhance accuracy, convergence speed, and computational robustness. The results establish a pathway toward reliable, high-fidelity PINN-based tools for structural deflection analysis and related inverse problems.

\section{Literature Review}
\subsection{Foundations of PINNs}
Physics-Informed Neural Networks (PINNs) integrate the governing physical laws of a system into the loss function of a neural network, enabling the solution of partial differential equations (PDEs) without explicit discretization. The foundational work by \citet{Raissi2019} demonstrated the embedding of physics constraints into neural network training, leveraging automatic differentiation \citep{Baydin2018} to evaluate derivatives efficiently. The general PINN formulation addresses problems of the form:
\[
\mathcal{N}[u(\mathbf{x})] = f(\mathbf{x}), \quad \mathcal{B}[u(\mathbf{x})] = g(\mathbf{x}),
\]
with a composite loss:
\[
\mathcal{L} = w_{\text{PDE}} \mathcal{L}_{\text{PDE}} + w_{\text{BC}} \mathcal{L}_{\text{BC}}.
\]

\subsection{Advancements in Methodologies}
Research has addressed PINNs' convergence and stability issues through adaptive weighting \citep{Wang2021}, improved initialization schemes \citep{Lu2021}, and spectral embeddings for faster convergence \citep{Abueidda2021}. Gaussian regularization for singularity handling has been optimized \citep{Hao2022}, and large-scale implementations using domain decomposition have been explored \citep{Peng2023}.

\subsection{Structural Engineering Applications}
PINNs have been applied to beam deflection \citep{Zhang2020}, Timoshenko beam models \citep{Samaniego2020}, and plate bending \citep{Abueidda2022}. Extensions include frame systems with semi-rigid connections \citep{Niaki2021} and welded joints stress prediction \citep{Gao2022}. For inverse problems, applications include load estimation on bridges \citep{Fuhg2021} and material damage identification \citep{Sun2022}.

\subsection{Algorithmic Innovations}
Constraint enforcement is achieved via soft penalty-based methods \citep{Raissi2019} or hard analytical transformations \citep{Lu2021}. Adaptive schedulers \citep{McClenny2022} improve convergence. For singularities, residual-based adaptive sampling \citep{Sharma2022} and inequality constraint formulations \citep{Guo2023} have been effective. Multi-scale frameworks coupling PINNs with FEM have been developed \citep{Hughes2022}.

\subsection{Validation and Benchmarks}
Benchmark studies \citep{Kollmannsberger2021} show sub-percent errors for various beam configurations. Convergence analyses \citep{Haghighat2023} highlight the role of activation functions and initializations, while computational efficiency improvements over FEM have been quantified \citep{Berghoff2023}.

\subsection{Challenges and Directions}
Key challenges include modeling nonlinear materials \citep{Mozaffar2022}, composite delamination \citep{Bessa2023}, and large deformations \citep{Viana2024}. Emerging areas involve operator learning \citep{Li2023}, quantum-enhanced PINNs \citep{Abu-Mostafa2024}, and digital twin frameworks \citep{Ikeda2024}. Table~\ref{tab:challenges} summarizes selected challenges and proposed solutions.

\begin{table}[h!]
\centering
\begin{tabular}{ll}
\toprule
\textbf{Challenge} & \textbf{Emerging Solution} \\
\midrule
High-order continuity & B-spline enriched networks \citep{Shen2024} \\
Experimental noise & Physics-regularized filters \citep{Pati2023} \\
3D scalability & Hybrid FEM–PINN solvers \citep{Zhang2024} \\
\bottomrule
\end{tabular}
\caption{Summary of challenges and emerging solutions in PINN research.}
\label{tab:challenges}
\end{table}

This review highlights the rapid evolution of PINNs, particularly in structural engineering, and underscores the need for robust, scalable, and physically consistent models to address the next generation of engineering problems.


\section{Methodology}
This section details the development of the proposed Physics-Informed Neural Network (PINN) framework for Euler–Bernoulli beam deflection analysis. The methodology integrates three main innovations: hard-constrained output transformation for exact enforcement of high-order boundary conditions, adaptive loss weighting for balanced optimization, and Gaussian-regularized Dirac delta representation for concentrated loads.

\subsection{Governing Equation}
The Euler–Bernoulli beam model is governed by:
\begin{equation}
EI \frac{\mathrm{d}^4 w(x)}{\mathrm{d}x^4} = q(x),
\label{eq:euler-bernoulli}
\end{equation}
where $E$ is Young’s modulus, $I$ is the moment of inertia, $w(x)$ is the deflection, and $q(x)$ is the transverse load distribution. Boundary conditions depend on the beam configuration (e.g., clamped, simply supported).

\subsection{Hard-Constrained Output Transformation}
To exactly satisfy boundary conditions for fourth-order PDEs, we employ an analytical transformation:
\begin{equation}
\hat{w}_\theta(x) = T(x) \cdot N_\theta(x),
\end{equation}
where $T(x)$ is a polynomial factor encoding boundary constraints and $N_\theta(x)$ is the neural network output. For a clamped–clamped beam of length $L$, $T(x) = x^2 (L-x)^2$ ensures $w(0) = w'(0) = w(L) = w'(L) = 0$.

\subsection{Adaptive Boundary Loss Weighting}
The total loss is:
\begin{equation}
\mathcal{L} = w_{\text{PDE}}(t) \mathcal{L}_{\text{PDE}} + w_{\text{BC}}(t) \mathcal{L}_{\text{BC}},
\end{equation}
with $w_{\text{BC}}(t) = w_0 \exp(-\alpha t)$, where $w_0$ is the initial boundary loss weight and $\alpha$ controls decay rate. This scheme balances PDE residuals and boundary conditions over training iterations.

\subsection{Gaussian-Regularized Dirac Delta Loads}
Point loads are modeled via Gaussian approximation:
\begin{equation}
\delta(x - x_0) \approx \frac{1}{\sqrt{2\pi}\sigma} \exp\left(-\frac{(x - x_0)^2}{2\sigma^2}\right),
\end{equation}
with $\sigma = 0.01L$ to localize the load without introducing discontinuities.

\subsection{Neural Network Architecture and Training}
The network $N_\theta$ consists of 6 hidden layers with 40 neurons each, \texttt{tanh} activation, and Xavier initialization. Optimization is performed in two stages: Adam optimizer for 50,000 iterations (learning rate $10^{-3}$), followed by L-BFGS until convergence. Automatic differentiation is used to compute PDE residuals and their derivatives.

\subsection{Implementation}
All models are implemented in TensorFlow 2.12, with experiments conducted on an RTX 3050 GPU. Training and evaluation scripts are available in the supplementary repository.

This methodological framework enables precise boundary condition enforcement, efficient convergence, and accurate handling of singular loading scenarios for beam deflection analysis.


\section{Applications}
This section presents three representative beam deflection problems to evaluate the proposed PINN framework: (1) a cantilever beam under uniform distributed load, (2) a fully restrained beam under uniform load, and (3) a simply supported beam subjected to a concentrated mid-span point load. Each case illustrates the adaptability of the methodology to different boundary conditions and loading types.

\subsection{Case 1: Cantilever Beam under Uniform Load}
A cantilever beam of length $L$ is subjected to a uniform load $q_0$. The governing PDE follows Eq.~\eqref{eq:euler-bernoulli}, with boundary conditions $w(0) = w'(0) = 0$ at the clamped end and $w''(L) = w'''(L) = 0$ at the free end. The output transformation $T(x) = x^2$ enforces the clamped conditions exactly. The analytical solution is used for validation.

\subsection{Case 2: Fully Restrained Beam under Uniform Load}
A beam clamped at both ends is loaded uniformly along its span. Boundary conditions $w(0) = w'(0) = w(L) = w'(L) = 0$ are enforced via $T(x) = x^2(L-x)^2$. The adaptive boundary loss weighting accelerates convergence by initially prioritizing boundary satisfaction before shifting focus to PDE residual minimization.

\subsection{Case 3: Simply Supported Beam with Mid-Span Point Load}
A simply supported beam of length $L$ is subjected to a concentrated load $P$ at $x = L/2$. The Dirac delta in the governing equation is approximated using a Gaussian kernel with $\sigma = 0.01L$, enabling smooth residual computation without domain splitting. Boundary conditions $w(0) = w(L) = 0$ and $w''(0) = w''(L) = 0$ are imposed through $T(x) = x(L-x)$.

\subsection{Performance Metrics}
For each case, accuracy is quantified using the relative $L^2$ error, mean absolute error (MAE), and root mean square error (RMSE) against the analytical or FEM solution. Training efficiency is assessed through convergence rate and loss component evolution.

\subsection{Observations}
The PINN framework achieved sub-percent relative $L^2$ errors in all cases. The Gaussian regularization significantly improved stability for point load scenarios, while the adaptive loss weighting enhanced convergence speed for all boundary configurations. These results confirm the robustness and versatility of the proposed approach for diverse beam deflection problems.


\section{Results and Discussion}
This section presents the quantitative and qualitative results obtained from the three beam deflection case studies, along with an analysis of training dynamics, accuracy, and computational performance.

\subsection{Quantitative Accuracy}
Table~\ref{tab:results} summarizes the performance metrics for each case. The proposed PINN framework achieved relative $L^2$ errors below 1\% in all configurations, with the lowest error (0.32\%) observed in the fully restrained beam case. MAE and RMSE values were similarly low, indicating consistent accuracy across the domain.

\begin{table}[h!]
\centering
\begin{tabular}{lccc}
\toprule
\textbf{Case} & $L^2$ Error (\%) & MAE & RMSE \\
\midrule
Cantilever, uniform load & 0.54 & $2.1\times 10^{-5}$ & $3.5\times 10^{-5}$ \\
Fully restrained, uniform load & 0.32 & $1.6\times 10^{-5}$ & $2.8\times 10^{-5}$ \\
Simply supported, point load & 0.56 & $2.4\times 10^{-5}$ & $3.9\times 10^{-5}$ \\
\bottomrule
\end{tabular}
\caption{Performance metrics for the three beam deflection problems.}
\label{tab:results}
\end{table}

\subsection{Training Dynamics}
Figure~\ref{fig:loss_evolution} shows the evolution of PDE and boundary condition losses over training iterations. The adaptive boundary loss weighting resulted in rapid satisfaction of boundary constraints in the early stages, followed by focused PDE residual minimization. Convergence was typically achieved within 80,000 total iterations.

\subsection{Effect of Gaussian Regularization}
For the point load case, Gaussian regularization prevented instability and ensured smooth residual gradients. Without regularization, convergence was slower and final errors were approximately twice as high.

\subsection{Comparison with FEM}
Comparison against finite element method (FEM) solutions showed strong agreement, with displacement profiles nearly overlapping (Figure~\ref{fig:beam_profiles}). PINNs offered mesh-independent performance, while FEM required mesh refinement to achieve similar accuracy.

\subsection{Computational Performance}
Training times ranged from 45 to 65 seconds on an RTX 3050 GPU, with memory usage remaining below 2 GB. The method's efficiency suggests potential for scaling to more complex geometries and 2D/3D structural models.

\subsection{Discussion}
The results demonstrate that the proposed PINN enhancements—hard-constrained transformations, adaptive loss weighting, and Gaussian-regularized singular loads—collectively improve convergence, stability, and accuracy. The framework is robust across varying boundary conditions and load types, making it a promising alternative to conventional numerical solvers for beam deflection problems.


\section{Results and Discussion}
This section presents the quantitative and qualitative results obtained from the three beam deflection case studies, along with an analysis of training dynamics, accuracy, and computational performance.

\subsection{Quantitative Accuracy}
Table~\ref{tab:results} summarizes the performance metrics for each case. The proposed PINN framework achieved relative $L^2$ errors below 1\% in all configurations, with the lowest error (0.32\%) observed in the fully restrained beam case. MAE and RMSE values were similarly low, indicating consistent accuracy across the domain.

\begin{table}[h!]
\centering
\begin{tabular}{lccc}
\toprule
\textbf{Case} & $L^2$ Error (\%) & MAE & RMSE \\
\midrule
Cantilever, uniform load & 0.54 & $2.1\times 10^{-5}$ & $3.5\times 10^{-5}$ \\
Fully restrained, uniform load & 0.32 & $1.6\times 10^{-5}$ & $2.8\times 10^{-5}$ \\
Simply supported, point load & 0.56 & $2.4\times 10^{-5}$ & $3.9\times 10^{-5}$ \\
\bottomrule
\end{tabular}
\caption{Performance metrics for the three beam deflection problems.}
\label{tab:results}
\end{table}

\subsection{Training Dynamics}
Figure~\ref{fig:loss_evolution} shows the evolution of PDE and boundary condition losses over training iterations. The adaptive boundary loss weighting resulted in rapid satisfaction of boundary constraints in the early stages, followed by focused PDE residual minimization. Convergence was typically achieved within 80,000 total iterations.

\subsection{Effect of Gaussian Regularization}
For the point load case, Gaussian regularization prevented instability and ensured smooth residual gradients. Without regularization, convergence was slower and final errors were approximately twice as high.

\subsection{Comparison with FEM}
Comparison against finite element method (FEM) solutions showed strong agreement, with displacement profiles nearly overlapping (Figure~\ref{fig:beam_profiles}). PINNs offered mesh-independent performance, while FEM required mesh refinement to achieve similar accuracy.

\subsection{Computational Performance}
Training times ranged from 45 to 65 seconds on an RTX 3050 GPU, with memory usage remaining below 2 GB. The method's efficiency suggests potential for scaling to more complex geometries and 2D/3D structural models.

\subsection{Discussion}
The results demonstrate that the proposed PINN enhancements—hard-constrained transformations, adaptive loss weighting, and Gaussian-regularized singular loads—collectively improve convergence, stability, and accuracy. The framework is robust across varying boundary conditions and load types, making it a promising alternative to conventional numerical solvers for beam deflection problems.

\section{Conclusions}
This work introduced a Physics-Informed Neural Network framework tailored for Euler–Bernoulli beam deflection analysis, incorporating three methodological advancements: (1) hard-constrained output transformations for exact enforcement of high-order boundary conditions, (2) adaptive boundary loss weighting for improved convergence balance, and (3) Gaussian-regularized Dirac delta representation for stable handling of concentrated loads. 

Extensive evaluations on cantilever, fully restrained, and simply supported beams demonstrated sub-percent relative $L^2$ errors, rapid convergence, and strong agreement with analytical and FEM solutions. The framework showed particular strength in stability for singular load cases and efficiency across different boundary configurations.

These findings position the proposed approach as a viable, mesh-independent alternative to traditional solvers, with potential extensions to nonlinear materials, 2D/3D geometries, and inverse problem settings. Future work will explore coupling with operator learning methods and applying the methodology to large-scale structural systems and real-time digital twins.


\bmhead{Supplementary Information}
Code repository available at: \url{https://github.com/Kiarash999/Physics-Informed-Neural-Networks-for-Beam-Deflection-Analysis--Methodology-and-Applications}

\bmhead{Acknowledgments}
The authors would like to thank Prof. Mirzabozorg and Prof. Mohtasham Dolatshahi for their guidance.

\bibliography{main-bibfile}
